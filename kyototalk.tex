\documentclass{beamer}



%\usepackage{beamerthemesplit}
\usetheme{Boadilla}
%\usetheme{default}
%\useinnertheme{rounded}

%\useoutertheme{shadow}
\usecolortheme{rose}
%\usefonttheme{serif}
\setbeamertemplate{navigation symbols}{}
\usetheme{Madrid}

\usepackage{amssymb,amsmath,amscd,amsfonts,amsthm,dsfont,color,graphicx}
\usepackage{amscd}
%\usepackage[numbers]{natbib}
% \usepackage[french]{babel}
%\usepackage[active]{srcltx}


% \date[]{}

 \newcommand\makebeamertitle{\frame{\maketitle}}%

 \AtBeginDocument{
   \let\origtableofcontents=\tableofcontents
   \def\tableofcontents{\@ifnextchar[{\origtableofcontents}{\gobbletableofcontents}}
   \def\gobbletableofcontents#1{\origtableofcontents}
 }
\numberwithin{equation}{section}
  \theoremstyle{plain}
  \newtheorem*{thm*}{\protect\theoremname}
  \theoremstyle{plain}
  \newtheorem*{cor*}{\protect\corollaryname}
 \theoremstyle{definition}
 \newtheorem*{defn*}{\protect\definitionname}
 \theoremstyle{plain}
\newtheorem*{lem*}{\protect\lemmaname}
  \theoremstyle{plain}
  \newtheorem*{rem*}{\protect\remarkname}
   \theoremstyle{definition}
 \newtheorem*{prop*}{\protect\propositionname}

\usetheme{Madrid}

\makeatother

  \providecommand{\corollaryname}{Corollary}
  \providecommand{\definitionname}{Definitioninition}
  \providecommand{\theoremname}{Theorem}
   \providecommand{\lemmaname}{Lemma}
   \providecommand{\remarkname}{Remark}
   \providecommand{\propositionname}{Proposition}
   
   


\newcommand{\Rl}{\mathbb{R}}
\newcommand{\Cplx}{\mathbb{C}}
\newcommand{\Itgr}{\mathbb{Z}}
\newcommand{\Ntrl}{\mathbb{N}}
\newcommand{\Circ}{\mathbb{T}}
\newcommand{\Sb}{\mathbb{S}}
\newcommand{\Disc}{\mathbb{D}}
\newcommand{\Aff}{\mathbb{A}}

% The Caligraphic alphabet
\newcommand{\Ac}{\mathcal{A}}
\newcommand{\Bc}{\mathcal{B}}
\newcommand{\Cc}{\mathcal{C}}
\newcommand{\Dc}{\mathcal{D}}
\newcommand{\Ec}{\mathcal{E}}
\newcommand{\Fc}{\mathcal{F}}
\newcommand{\Gc}{\mathcal{G}}
\newcommand{\Hc}{\mathcal{H}}
\newcommand{\Ic}{\mathcal{I}}
\newcommand{\Jc}{\mathcal{J}}
\newcommand{\Kc}{\mathcal{K}}
\newcommand{\Lc}{\mathcal{L}}
\newcommand{\Mv}{\mathcal{M}}
\newcommand{\Nv}{\mathcal{N}}
\newcommand{\Oc}{\mathcal{O}}
\newcommand{\Pc}{\mathcal{P}}
\newcommand{\Qc}{\mathcal{Q}}
\newcommand{\Rc}{\mathcal{R}}
\newcommand{\Sc}{\mathcal{S}}
\newcommand{\Tc}{\mathcal{T}}
\newcommand{\Uc}{\mathcal{U}}
\newcommand{\Vc}{\mathcal{V}}
\newcommand{\Wc}{\mathcal{W}}
\newcommand{\Xc}{\mathcal{X}}
\newcommand{\Yc}{\mathcal{Y}}
\newcommand{\Zc}{\mathcal{Z}}


\newcommand{\Sp}{\mathrm{Sp}}
\newcommand{\tr}{\mathrm{tr}}
\newcommand{\Op}{\mathrm{Op}}
\newcommand{\sym}{\mathrm{sym}}
\newcommand{\Vol}{\mathrm{Vol}}
\newcommand{\Tr}{\mathrm{Tr}}
\newcommand{\dist}{\mathrm{dist}}
\newcommand{\sgn}{\operatorname{sgn}}
\newcommand{\diag}{\mathrm{diag}}
\newcommand{\id}{\mathrm{id}}
\newcommand{\Poly}{\mathrm{Poly}}

\newcommand{\spec}{\mathrm{Spec}}
\newcommand{\abs}{\mathrm{abs}}

\newcommand{\CV}{\mathrm{CV}}
\newcommand{\PCV}{\mathrm{PCV}}


% Used for highlighting. To remove all highlighting just make the command blank
\newcommand{\hl}{\color{red}}


\newcommand{\gf}{\mathfrak{g}}
\newcommand{\tf}{\mathfrak{t}}
\newcommand{\Str}{\mathrm{STr}}



\newcommand{\dom}{\mathrm{dom}}
\newcommand{\supp}{\mathrm{supp}}
\newcommand{\BS}{\mathfrak{BS}}
\newcommand{\loc}{\mathrm{loc}}
\newcommand{\re}{\mathrm{re}}
\newcommand{\im}{\mathrm{im}}

% DOI transformer
\newcommand{\Ti}{\mathcal{T}}


\def\Xint#1{\mathchoice
{\XXint\displaystyle\textstyle{#1}}%
{\XXint\textstyle\scriptstyle{#1}}%
{\XXint\scriptstyle\scriptscriptstyle{#1}}%
{\XXint\scriptscriptstyle\scriptscriptstyle{#1}}%
\!\int}
\def\XXint#1#2#3{{\setbox0=\hbox{$#1{#2#3}{\int}$ }
\vcenter{\hbox{$#2#3$ }}\kern-.6\wd0}}
\def\qint{\Xint-}

\def\qd{\,{\mathchar'26\mkern-12mu d}}


   
   
   
\begin{document}

\title[Introduction to the Hypoelliptic Laplacian]{Introduction to the Hypoelliptic Laplacian on a compact group}


\author[E. McDonald]{Ed McDonald\\
Based on joint work with N.~Higson, S.~Liu, F.~Sukochev and D.~Zanin}


\institute[]{Penn State University}

\makebeamertitle

\begin{frame}{Introduction}
  The \textbf{hypoelliptic Laplacian} of J.~M.~Bismut has an intimidating reputation.
  
  \pause
  But... where does it come from?\\
  \pause
  What is it good for?\\
  \pause
  And, most importantly, how do we prove things about it?
\end{frame}

\section{In the beginning...}


\section{Frenkel's formula}

\begin{frame}
  \huge{Section 1: Frenkel's formula}
\end{frame}

\begin{frame}
  Let $G$ be a connected, simply connected, compact group. There is the Casimir element, $\Delta_G \in \Uc(\gf).$
  
  What is $\Tr(e^{t\Delta_G})$?
  
  Some possible answers:
  \begin{enumerate}
    \item{} Spectral (Peter-Weyl) representation: we have
    \[
        \Tr(e^{t\Delta_G}) = \sum_{\lambda\in CR^*_+} \left(\prod_{\alpha>0} \frac{\langle \alpha,\rho+\lambda\rangle}{\langle \alpha,\rho\rangle}\right)^2e^{-tB(\lambda+2\rho,\lambda)}
    \]
    where $W_+$ is the weight lattice, $\prod_{\alpha>0}$ is the product over positive roots, $\rho$ is the Weyl element, and $B$ is the bilinear pairing on $\gf^*.$
    \item{} Minakshisundaram-Pleijel expansion: asymptotically, as $t\to 0,$ we have
    \[
        \Tr(e^{t\Delta_G}) \sim (4\pi t)^{-\frac{\mathrm{dim}(G)}{2}}(a_0+a_1t+a_2t^2+\cdots)
    \]
    where $a_0, a_1, $ etc. are determined by the geometry of $G.$
  \end{enumerate}
  Is there another (more insightful) representation?
\end{frame}

\begin{frame}{Jacobi identity}
  If $G = \Circ$ is the unit circle, the answer is \emph{yes}. We have
  \[
    \Tr(e^{t\Delta_{\Circ}}) = \mathrm{Vol}(\Circ)(4\pi t)^{-\frac12}\sum_{n\in \Itgr} e^{-\frac{1}{4t}n^2}.
  \]
  We can prove this via the Poisson summation formula and the fact that
  \[
    \Tr(e^{t\Delta_{\Circ}}) = \sum_{n\in \Itgr} e^{-tn^2}
  \]
  
  
  Is there something similar for a general Lie group?
\end{frame}

\begin{frame}{Frenkel's formula}
  It turns out that $\Tr(e^{t\Delta_G})$ is too difficult to compute. Instead, we can compute a ``shifted" version
  \[
    \Tr(\lambda(e^H)e^{t\Delta_G})
  \]
  where $H\in \tf,$ and $\lambda(e^H)$ is the left shift operator by $e^H\in G.$
  \begin{theorem}[Frenkel (1984)]
      If $H\in \tf$ is regular (i.e., $\alpha(H)\neq 0$ for all roots $\alpha$), then
      \[
          \Tr(\lambda(e^H)e^{t\Delta_G}) = \frac{\mathrm{Vol}(G)e^{4\pi^2t|\rho|^2}}{(4\pi t)^{\frac{\mathrm{dim}(G)}{2}}\sigma(H)}\sum_{\gamma \in CR} \left(\prod_{\alpha>0} \langle 2\pi \alpha,H+\gamma\rangle\right) e^{-\frac{1}{4t}|H+\gamma|^2} 
      \]
      where $\sigma$ is the Weyl denominator.
  \end{theorem}
\end{frame}

\begin{frame}{Proof of Frenkel's formula}
    Frenkel's formula is proved using the explicit spectral decomposition of $\Delta_G$ and the Weyl character formula. From these facts, we have
    \[
      \Tr(\lambda(e^H)e^{t\Delta_G}) = \frac{1}{\sigma(H)}\sum_{\lambda \in CR^*_+} \sum_{w \in W} \sgn(w)e^{w(\lambda+\rho)(H)}e^{-tB(\lambda+2\rho,\lambda)}
    \]
    Some rearrangement and Poisson's summation formula give the result.
\end{frame}


\begin{frame}
    \huge{Section 2: The Primordial History, or, the Berline-Vergne localisation formula}
\end{frame}

\begin{frame}{Section introduction}
  In this section, I will attempt to give a ``folkloric" story for the origins of the hypoelliptic Laplacian.
  
  I am strongly indebetted to the paper of Choi--Takhtajan (2021).
\end{frame}

\begin{frame}{The Hamiltonian-Lagrangian correspondence}
  In its most general terms, the path integral method in quantum mechanics relates traces of semigroups
  to integrals over loop space. 
  
  The way it is supposed to work is as follows: $X$ is a manifold, and $\Lc \in C^\infty(TX)$ is a ``Lagrangian". The corresponding
  ``Hamiltonian" $\Hc\in C^\infty(T^*X)$ is related to $\Lc$ by the Legendre transform 
  \[
    \Hc(x,p) = \sup_{X \in T_xX} (p(X)-\Lc(x,X)),\quad (x,p) \in T^*X.
  \]
\end{frame}

\begin{frame}{The Hamiltonian-Lagrangian correspondence}
  We are supposed to have something like the following:
  \begin{equation}
    \Tr(e^{-t\Hc(x,-i\partial)}) = \int_{LX} e^{-\int_0^t \Lc(\gamma(s),\gamma'(s))\,ds} \Dc \gamma.
  \end{equation}
  Here, $\Hc(x,-i\partial)$ is an operator on the Hilbert space $L_2(X)$ defined by some kind of quantisation of $\Hc,$ $LX$ is the space of loops, that is functions $S^1\to X,$ and ``$\Dc\gamma$" is some kind of measure on loop space.\\
  
  
  Both sides of this formula are problematic, but the left-hand-side is a bit more accessible. 
\end{frame}

\begin{frame}{The Hamiltonian-Lagrangian correspondence}
  One case (and one of the few cases that is well-understood) is for $\Delta_g,$ the Laplace operator on a compact Riemannian manifold
  \[
    \frac12\Delta_gu = \frac12\det(g)^{-\frac12}\partial_{\alpha}(g^{\alpha,\beta}\det(g)^{\frac12}\partial_{\beta}u)
  \]
  This is supposed to be $\Hc(x,-i\partial),$ where $\Hc(x,p) = \frac12 g^{\alpha,\beta}(x)p_{\alpha}p_{\beta}.$ The corresponding Lagrangian is 
  \[
    \Lc(x,X) = \frac12 g_{\alpha,\beta}(x)X^{\alpha}X^{\beta}
  \]
  and 
  \[
    \Tr(e^{\frac12 t\Delta_g}) = \int_{LX} e^{-\int_0^t \frac12 g_{\alpha,\beta}(\gamma(s))\dot{\gamma}^{\alpha}(s)\dot{\gamma}^\beta(s)\,ds}\, \Dc \gamma
  \]
  where the integral on the right-hand side is rigorously defined as a limit of natural discrete approximations. See Anderson-Driver.
\end{frame}

\begin{frame}{Berline-Vergne localisation}
  Suppose that we're trying to compute the integral of a differential form $\alpha$ over a manifold $Z,$ like
  \[
    \int_Z \alpha,\quad \alpha\in \Omega^{\mathrm{top}}(Z).
  \]
  There is a very effective trick for computing this integral if we have some $\Circ$-action: Assume that $\Circ$ acts
  on $Z,$ and $\alpha$ is invariant under this action. We have
  \begin{theorem}[Berline-Vergne localisation]
    \[
      \int_Z \alpha = \int_{Z^{0}} \frac{\alpha}{e(N(Z/Z^0))}
    \]
    where $Z^0$ is the fixed submanifold under the $\Circ$-action, and $e(N(Z/Z^0))$ is the Euler class of the normal bundle to $Z^0$ in $Z.$    
  \end{theorem}
  What is of greater interest to us is actually the standard proof of the Berline-Vergne formula.
\end{frame}

\begin{frame}{Proof of Berline-Vergne localisation}
    Berline-Vergne localisation can be proved in the following way. It is possible to construct a differential form $\Vc \in \Omega^2(X)$ such that
    \[
      \int_Z \alpha = \int_Z \alpha e^{b\Vc},\quad b>0.
    \]
    This is proved by showing that the right hand side is independent of $b,$ and then setting $b=0.$ The limit as $b\to\infty$ is computed using Laplace's method.    
\end{frame}

\begin{frame}{Bismut's dream}
  We would like to compute $\Tr(e^{t\Delta_g})$ by considering
  \[
    \int_{LX} e^{-\int_0^t \Lc(\gamma(s),\dot{\gamma}(s))\,ds} \Dc\gamma
  \]
  as an integral of a differential form over the infinite dimensional manifold $LX.$
  \pause
  We could try to mimic the proof of Berline-Vergne by introducing a $1$-parameter perturbation of $\Lc,$ something like
  \[
      \int_{LX} e^{-\int_0^t \Lc(\gamma(s),\dot{\gamma}(s))+b\Vc\,ds} \Dc\gamma
  \]
  where $b>0,$ and computing the limit as $b\to\infty.$
  \pause
  This is \textbf{TOO HARD}.
\end{frame}

\begin{frame}{Bismut's dream}
  Doing Berline-Vergne in loop space is too difficult, the technicalities in making sense of the path integral are too great. 
  
  But maybe, we can use the Hamiltonian-Lagrangian correspondence to replace incomprehensible loop space integrals with slightly more comprehensible operator traces.
  \begin{center}
    Can we find an operator $\Lc_b$ depending on a parameter $b$ such that
    \[
      \Tr(e^{t\Delta_g}) = \Tr(e^{t\Lc_b})
    \]
    for all $b>0,$ and such that the limit as $b\to\infty$ gives an interesting formula for the trace?
  \end{center}
  \textbf{Answer}: In a way, yes. Not precisely as stated above, but we can get close enough.
\end{frame}

\begin{frame}{The hypoelliptic Laplacian}
  Let $G$ be a compact Lie group, and let $H\in \tf$ be regular. Through some combination of path integral computations, intuition, and guesswork, Bismut found a solution the preceding question.
  \begin{theorem}
    There is a differential operator $\Lc_{b,H},$ depending on a positive real parameter $b,$ acting on the space
    \[
      C^\infty(\gf\times G,\bigwedge^{\bullet}\gf)
    \]
    such that
    \[
       \Tr(\lambda(e^H)e^{t\Delta_G}) = \Str(e^{-t\Lc_{b,H}}),\quad b>0.
    \]
    As $b\to\infty,$ it can be proved that $\Str(e^{-t\Lc_{b,H}})$ converges to Frenkel's formula.
  \end{theorem}
\end{frame}

\section{The form of $\Lc_b.$}

\begin{frame}
  \huge{The Form of $\Lc_b$}
\end{frame}

\begin{frame}{Notation}
  $G$ is a compact Lie group of dimension $d,$ and $\gf$ is its Lie algebra. Let $e_1,\ldots,e_d$ be an orthonormal basis of $\gf.$
  
  We will write the variables of $\gf\times G$ as $(y,x),$ and we have some clifford operators on $\bigwedge^{\bullet}\gf,$ denoted
  \[
    c(v) = v\wedge + \iota_v,\quad \widehat{c}(v) = v\wedge -\iota_v,\quad v \in \gf.
  \]
\end{frame}

\begin{frame}{Definition of $\Lc_b.$}
  \begin{definition}
    \[
      \Lc_b = (D+\frac1bQ)^2-D^2
    \]
  \end{definition}
  where
  \[
    Q = \sum_{j=1}^d c(e_j)y_j-i\widehat{c}(e_j)\partial_{y_j}
  \]
  and 
  \[
    D = \sum_{j=1}^d -ic(e_j)\partial_{x_j}+\frac{1}{12}f^j_{k,l}c(e_j)c(e_k)c(e_\ell)
  \]
\end{frame}

\begin{frame}{Structure of $\Lc_b.$}
  Expanding out the definitions, we get
  \[
    \Lc_b = \frac{1}{b^2}Q^2+\frac{1}{b}(QD+DQ)
  \]
  which turns out to be
  \[
    \Lc_b = \frac{1}{b^2}\sum_{j=1}^d y_j^2-\partial_{y_j}^2 + \frac{1}{b}\sum_{j=1}^dy_j\partial_{x_j} + \mathrm{matrix terms.}
  \]
\end{frame}

\section{Analytic problems and their solutions}

\begin{frame}
  \huge{Analytic problems and their solutions}
\end{frame}

\begin{frame}{The problems we face}
  The main analytic difficulties we need to deal with are proving \emph{Schauder estimates}
  \[
    \|u\|_{s+2} \lesssim \|\Lc_bu\|_s+\|u\|_s,\quad \|v\|_{s+2} \lesssim \|(\partial_t-\Lc_b)v\|_s+\|v\|_s
  \]
  and \emph{heat kernel estimates}
  \[
    \exp(-t\Lc_b)(p,q) \leq \text{(???)}
  \]
\end{frame}


\section{The world's most convoluted proof of Frenkel's formula}

\begin{frame}
  \huge{Section 4: The world's most convoluted proof of Frenkel's formula}
\end{frame}

\subsection{As $b\to 0.$}

\begin{frame}{The limit as $b\to 0.$}
    In some sense, we are supposed to have
    \[
      \exp(-t\Lc_b)\rightarrow \exp(t\Delta_G).
    \]
    But how does this actually work? After all, they act on different spaces.
\end{frame}

\begin{frame}{The projection onto $\ker(Q).$}
  In order to understand how an operator on $L_2(\gf\times G,\bigwedge^{\bullet}\gf)$ converges to an operator on
  $L_2(G),$ we need to understand how $L_2(G)$ is a subspace of $L_2(\gf\times G,\bigwedge^{\bullet}\gf).$
  
  \textbf{Solution}: we use a Gaussian function on the $\gf$ fibre. 
  This is like embedding $L_2(\Circ)$ isometrically into $L_2(\Circ\times \Rl)$ by mapping $f(x)$ to $f(x)e^{-\pi y^2}.$
  
  \begin{definition}
    Recall that $Q$ is the Witten Dirac,
    \[
      Q = \sum_{j=1}^d c(e_j)y_j - \widehat{c}(e_j)\partial_{y_j}.
    \]
    Let $P$ be the kernel projection of $Q.$
  \end{definition}
  We will embed $L_2(G)$ into $L_2(\gf\times G,\bigwedge^{\bullet}\gf)$ by
  \[
    L_2(G)\approx  PL_2(\gf\times G,\bigwedge^{\bullet}\gf)P.
  \]
\end{frame}

\begin{frame}{The $2\times 2$ matrix trick}
  We want to prove that $\exp(-t\Lc_b)$ converges to $P\exp(t\Delta_G)P.$ Actually it is simpler to work with resolvents, and prove that
  \[
    (\lambda+\Lc_b)^{-1} \rightarrow P(\lambda-\Delta_G)^{-1}P
  \]
  for a suitably large set of $\lambda\in \Cplx.$ By some functional calculus, this is equivalent. 
  
  \textbf{Idea:} write the resolvent as a $2\times 2$ matrix.
\end{frame}

\subsection{As $b\to \infty.$}

\begin{frame}
  \huge{As $b\to\infty$}
\end{frame}

\begin{frame}
  The much more challenging problem we face is to understand what happens to $\Str(e^{-t\Lc_b})$ as $b\to\infty.$
  Note that:
  \begin{enumerate}
    \item{} Unlike with $b\to 0,$ we have to take the supertrace first in order that the limit exists. This is analogous to the local index theorem. 
    \item{} This argument will be complicated, in part, because we have to do a simultaneous rescaling of the spatial and clifford variables.
    \item{} It is not possible to compute the limit of $\Str(e^{-t\Lc_b})$ directly, instead we need to introduce a shift.
  \end{enumerate}
\end{frame}

\begin{frame}{Techniques}
  Going into all the details with the rescaling of variables is too much to attempt here.... Instead I will briefly indicate what we're aiming for.
  
  What we expect is that
  \[
    \lim_{b\to\infty}\Str(e^{-t\Lc_{b,H}}) = \text{A sum over coweights of }G.
  \]
  The left hand side is an integral over the kernel of the operator. How does an integral converge to a sum?
  
\end{frame}

\section{Conclusions}

\begin{frame}{What next?}
  We have succeeded in proving Frenkel's formula by replacing some elementary representation theory and Poisson's summation formula by the most convoluted arguments imaginable. \\
  
  We can all agree that having new proofs of old results is worthwhile in itself, but...
  \textbf{What is the point of all this?}
  \begin{enumerate}
    \item{} Bismut's proof is \emph{totally different} to Frenkel's proof. Strikingly, \emph{no representation theory is needed}. We didn't even need the theorem of the highest weight. The coweight lattice turned up in the sum for completely geometric reasons.
    \item{} Bismut eventually extended his arguments to give totally new results, including his formulas for the heat kernel on symmetric spaces.
  \end{enumerate}
\end{frame}


\begin{frame}
\structure{\begin{center}
{\huge{}Thank you for listening!}\\
\begin{center}
{\Huge{}Happy Birthday Nigel!}
\end{center}
\par\end{center}}\end{frame}



\end{document}

